%% hf: enable header and footer.
\documentclass[
% hf, % <-- Not sure if this should be enabled
]{ceurart}

% One can fix some overfills
% \sloppy

% This was used to mitigate some warnings
\usepackage[T1]{fontenc}
\usepackage{lmodern}
% Modern tables + extra stuff workaround
\selectcolormodel{natural}
\usepackage{ninecolors}
\selectcolormodel{rgb}
\usepackage{tabularray}

\begin{document}

% Rights management information. CC-BY is the default license.
\copyrightyear{2024}
\copyrightclause{Copyright for this paper by its authors.
  Use permitted under Creative Commons License Attribution 4.0
  International (CC BY 4.0).}

\conference{BPM 2024: Demos and Resources, September 01-06, 2024, Krakow, PL}

\title{BPMN Analyzer 2.0: Instantaneous, Comprehensible, and Fixable Control Flow Analysis for BPMN Models}
% TODO: Update the DOI link to fit the extended version on arxiv

\author[1]{Tim Kräuter}
[email=tkra@hvl.no]
\author[1]{Patrick Stünkel}
[email=past@hvl.no] % patrick.stuenkel@hvl.no
\author[1]{Adrian Rutle}
[email=aru@hvl.no]
\author[1]{Yngve Lamo}
[email=yla@hvl.no]
\author[2,1]{Harald König}
[email=harald.koenig@fhdw.de]
\address[1]{Western Norway University of Applied Sciences, Bergen, Norway}
\address[2]{FHDW Hannover, Germany}

\begin{abstract}
% 102 words
Many business process models contain control flow errors, such as deadlocks or livelocks, which hinder proper execution.
In this paper, we introduce a new tool that can instantaneously identify control flow errors in BPMN models, make them understandable for modelers, and suggest corrections to resolve them.
We demonstrate that detection is instantaneous by benchmarking our tool against synthetic BPMN models with increasing size and state space complexity, as well as realistic models.
Moreover, the tool directly displays detected errors in the model, including an interactive visualization, and suggests fixes to resolve them.
The tool is open-source, extensible, and integrated into a popular BPMN modeling tool.
\end{abstract}

\begin{keywords}
BPM,
Verification,
Control flow analysis,
BPMN model checking,
Soundness,
Safeness
\end{keywords}

\maketitle
% 5 pages including references

\section{Introduction}

% Problem statement
Business Process Modeling Notation (BPMN) is becoming increasingly popular for automating processes and orchestrating people and systems.
However, many process models suffer from control flow errors, such as deadlocks, livelocks, starvation, and lack of synchronization~\cite{fahlandAnalysisDemandInstantaneous2011}.
These errors hinder the correct execution of BPMN models and may be detected late in the development process, resulting in elevated costs.

% Solution
In this paper, we describe a new tool, the \textit{BPMN Analyzer 2.0}\footnote{
In the following, we will use BPMN Analyzer to refer to the BPMN Analyzer 2.0, not our previous BPMN Analyzer described in~\cite{krauterFormalizationAnalysisBPMN2023}.}, for analyzing BPMN process models to detect control flow errors \textit{already} during modeling. % TODO: Maybe give concrete examples of what we can check!
\autoref{fig:overview} shows an overview of the tool.
The tool front-end is based on the popular \textit{bpmn.io} ecosystem, while the analysis is implemented in Rust for performance and memory efficiency reasons.
The BPMN Analyzer is open-source and accessible online\footnote{\url{https://timkraeuter.com/bpmn-analyzer-js/}} alongside a video demonstration\footnote{\url{https://www.youtube.com/watch?v=MxXbNUl6IjE}}~\cite{krauterInstantaneousComprehensibleFixable2024}.

\begin{figure}[ht]
	\centering
	\includegraphics[width=0.6\linewidth]{images/overview}
	\caption{Overview of the BPMN Analyzer 2.0}
	\label{fig:overview}
\end{figure}

The tool can check models after each change since analysis is \textit{instantaneous} according to~\cite{fahlandAnalysisDemandInstantaneous2011}, i.e., it takes 500ms or less.
Furthermore, we ensure the results are \textit{comprehensible} by highlighting possible violations directly in the model and displaying an interactive counterexample visualization.
Finally, the tool suggests \textit{fixes} for the most common soundness violations and can be extended to suggest more fixes in the future.

% Why is it significant for the BPM field?
Fahland et al.~\cite{fahlandAnalysisDemandInstantaneous2011} describe \textit{coverage}, \textit{immediacy}, and \textit{consumability} as the main challenges for users unaccustomed to formal analysis methods.
The BPMN Analyzer addresses all these challenges since it supports the most common BPMN elements used in practice (coverage), provides \textit{instantaneous} results (immediacy), and a \textit{comprehensible} user interface (consumability), even including suggestions of fixes.
Developers of industrial BPMN solutions also like our tool, especially the End-2-End user journey~\cite{krauterInstantaneousComprehensibleFixable2024}.
Thus validating our claim of a UI that is understandable for users unfamiliar with formal analysis methods.

% Paper structure
In the remainder of the paper, we describe how instantaneous, comprehensible, and fixable control flow error detection is achieved in \autoref{sec:innovations}.
Then, we discuss the maturity of our tool in \autoref{sec:maturity} before presenting related work in \autoref{sec:related-work}.
Finally, we conclude in \autoref{sec:conclusion}.

% Must have section according to the description
% A section discussing the innovations of the tool or resource to the BPM community and its main characteristics or features
\section{Innovations} \label{sec:innovations} % Main tool description and showing the innovations
The BPMN Analyzer has three main innovations: \textbf{instantaneous}, \textbf{comprehensible}, and \textbf{fixable} control flow error detection.
In this section, we will present the innovations, and more details can be found in our extended paper~\cite{krauterInstantaneousComprehensibleFixable2024}.

\subsection{Instantaneous Analysis}

We demonstrate instantaneous control flow analysis by benchmarking our tool in \textit{three} different scenarios.
For all our benchmarks, we use the hyperfine benchmarking tool (version 1.18.0), which calculates the average runtime when executing each control flow analysis ten or more times.
The benchmarks were run on Ubuntu 22.04.4 with an AMD Ryzen 7700X processor (4.5GHz) and 32 GB of RAM (5600 MHz).
All used BPMN models, our tools to generate them, and benchmarking scripts to run them are available in~\cite{krauterInstantaneousComprehensibleFixable2024}.

\textbf{First}, we benchmarked how our tool handles \textbf{BPMN models of growing size}.
We generated 500 synthetic BPMN models starting with five elements up to 4000 elements, increasing linearly in size.
The models repeatedly contain three activities and an exclusive/parallel block with two branches containing one activity per branch (more details in~\cite{krauterInstantaneousComprehensibleFixable2024}).
The BPMN Analyzer spends from 1 ms up to 9 ms for the BPMN models compared to 0.7 s up to 14 s in our previous tool~\cite{krauterFormalizationAnalysisBPMN2023}.
In summary, the linear growth of the models leads to a linear growth in state space and, consequently, a linear growth in runtime.

\textbf{Second}, we benchmarked the tool against a synthetic data set of models that lead to a state space explosion.
This represents a \textit{worst case} scenario for formal analysis.
We generated a data set of models~\cite{krauterInstantaneousComprehensibleFixable2024} with a growing number of parallel branches with increasing length, similar to~\cite{corradiniFormalApproachAnalysis2021}.
\autoref{tab:parallel-branches-benchmark} shows the average runtime of our tool when analyzing these models.
The BPMN Analyzer explores the entire state space while simultaneously analyzing the control flow, i.e., verifying soundness properties.

\begin{table}[h]
	\centering
	\caption{Benchmark results for models with varying parallel branches}
	\label{tab:parallel-branches-benchmark}
	\SetTblrInner{colsep=2pt}
	\begin{tblr}{
			column{1-X} = {c},
			column{Y-Z} = {r},
			hline{1, 2, Z} = {-}{1.2pt, solid}, % Z is the last row/column
			hline{6, 7} = {-}{dashed},
			vline{2-Y} = {2-Z}{solid}, % Y is the second to last row/column
		}
		\textbf{Branches} & \textbf{Branch Length} &\textbf{Runtime} & \textbf{States} \\
		15 & 1 & 161 ms& 32.771 \\
		16 & 1 & 360 ms& 65.539 \\
		17 & 1 & 790 ms& 131.075 \\
		20 & 1 & 8.803 ms& 1.048.579 \\
		5 & 5 & 14 ms& 7.779 \\
		3 & 20 & 11 ms& 9.264 \\
	\end{tblr}
\end{table}

The models' state space grows exponentially, leading to the same order of growth in runtime.
Our analysis is not instantaneous anymore when approaching 17 parallel branches of length 1 (see \autoref{tab:parallel-branches-benchmark}).
However, analysis is still instantaneous for more reasonable models with five parallel branches of length 5 or 3 branches of length 20.
Other tools report 2-3s of runtime for most soundness properties and 30s for a model with five parallel branches~\cite{corradiniFormalApproachAnalysis2021}, which took us only milliseconds.

\textbf{Third}, we applied our tool to eight \textbf{realistic models}, where three models (e001, e002, e020) are taken from~\cite{houhouFirstOrderLogicVerification2022}, and the remaining five models are part of the Camunda BPMN for research repository\footnote{\url{https://github.com/camunda/bpmn-for-research} \label{footnote:camundaResearch}}.
The BPMN Analyzer takes 1-10ms for e001, e002, and e020 ~\cite{krauterInstantaneousComprehensibleFixable2024} while~\cite{houhouFirstOrderLogicVerification2022} and~\cite{krauterFormalizationAnalysisBPMN2023} report 3.66-10.26s and 1-1.75s.
The benchmarks in~\cite{krauterFormalizationAnalysisBPMN2023} were run on the same hardware, while the machine used in~\cite{houhouFirstOrderLogicVerification2022} was slightly less powerful.
To summarize, we believe our analysis is instantaneous for nearly all models since most models have relatively small state spaces (less than 1000 states, see~\cite{fahlandAnalysisDemandInstantaneous2011}).

\subsection{Comprehensible Analysis}
We implemented two features to make control flow analysis understandable for everyone. 
% Highlighting
\textbf{First}, we highlight the problematic elements that cause control flow errors by directly attaching red overlays to them in the BPMN model.
In addition, there is a summary panel stating which if any errors are found.

% TODO: A screenshot of the tool would be nice here!

% Counterexample
\textbf{Second}, we use \textit{tokens} to \textit{interactively} visualize errors, i.e., we show an execution leading to the error.
Our analysis provides sample executions leading to the possible control-flow errors, which we visualize directly in the BPMN editor by showing how tokens move from the process start to an erroneous state.
We are unaware of other tools that visualize errors directly in the editor and allow for interactions, such as stopping/resuming, restarting, and speeding up the execution.

\subsection{Fixable Analysis}
In addition to detecting, highlighting, and visualizing control flow errors, the BPMN Analyzer suggests fixes similar to \textit{quick fixes} in Integrated Development Environments.
Quick fixes cannot be provided for all errors, but we currently cover some patterns leading to deadlocks, lack of synchronization, message starvation, and reused end events.
The quick fixes we support are described in detail in~\cite{krauterInstantaneousComprehensibleFixable2024} and can be extended independently of the formal analysis.
We are unaware of other tools that offer fixes for identified control-flow errors.

% They also want this section
\section{Maturity of the tool} \label{sec:maturity}
The BPMN Analyzer is our newest tool, incorporating many findings from our previous work~\cite{krauterFormalizationAnalysisBPMN2023} while focusing on instantaneous and understandable error detection.
The tool is open-source~\cite{krauterInstantaneousComprehensibleFixable2024}, and we aim to achieve high code quality by employing industry best practices such as rigorous static analysis, testing, and consistent formatting.
Furthermore, we demonstrated the tool to companies in the BPMN process orchestration space and received positive initial feedback, as described in the introduction.

% Scalability tests (instantaneous) --> Generated models are public, including our tool for future direct comparisons
As described in the previous section, we benchmarked our tool runtime in three scenarios.
The tool can check most models instantaneously, and we provide more details, including scripts and data needed to reproduce our results in~\cite{krauterInstantaneousComprehensibleFixable2024}.

\section{Related work} \label{sec:related-work}

BPMN analysis tools or formalizations are typically compared by their \textbf{coverage} of BPMN elements~\cite{krauterFormalizationAnalysisBPMN2023,corradiniFormalApproachAnalysis2021,houhouFirstOrderLogicVerification2022}.
Our tool already supports the common BPMN elements~\cite{krauterInstantaneousComprehensibleFixable2024}, and we aim for coverage similar to popular BPMN process orchestration platforms such as Camunda.
% Breakdown of BPMN coverage: https://github.com/timKraeuter/rust_bpmn_analyzer?tab=readme-ov-file#bpmn-coverage
However, we agree with~\cite{fahlandAnalysisDemandInstantaneous2011} that \textbf{analysis runtime} and \textbf{understandability} are the main issues that hinder the wide adoption of formal analysis tools outside academia.

Our tool has comparable BPMN element coverage to~\cite{corradiniFormalApproachAnalysis2021} but is still missing some elements when compared to~\cite{krauterFormalizationAnalysisBPMN2023}, which are currently on the roadmap. 
In addition, we demonstrated the instantaneous analysis of the BPMN Analyzer, which outperforms other tools~\cite{krauterFormalizationAnalysisBPMN2023,corradiniFormalApproachAnalysis2021,houhouFirstOrderLogicVerification2022} as described in \autoref{sec:innovations}.
Finally, we have only received positive feedback regarding the tool's usability.

\section{Conclusion} \label{sec:conclusion}

In this paper, we describe the novel \textit{BPMN Analyzer} that provides instantaneous, comprehensible, and fixable BPMN control flow error detection and is integrated into a popular BPMN modeling tool.
We benchmarked our tool against synthetic and realistic BPMN models to demonstrate instantaneous soundness checking.
We address the three main challenges: \textit{coverage}, \textit{immediacy}, and \textit{consumability} for providing formal analysis to non-expert users as identified in~\cite{fahlandAnalysisDemandInstantaneous2011}.
In addition, our tool provides quick fixes for common patterns leading to control flow errors.
One can understand the BPMN Analyzer as a BPMN-specific model checker, implemented in Rust paired with an intuitive user interface based on the popular \textit{bpmn.io} ecosystem that allows easy extension in the future.

% \begin{acknowledgments}
% Thanks to the anonymous reviewers for their valuable feedback.
% \end{acknowledgments}

\bibliography{bib}

\end{document}
