%% hf: enable header and footer.
\documentclass[
twocolumn,
% hf, % <-- Not sure if this should be enabled
]{ceurart}

% One can fix some overfills
% \sloppy

% This was used to mitigate some warnings
\usepackage[T1]{fontenc}
\usepackage{lmodern}
% Modern tables + extra stuff workaround
\selectcolormodel{natural}
\usepackage{ninecolors}
\selectcolormodel{rgb}
\usepackage{tabularray}

\begin{document}

% Rights management information. CC-BY is the default license.
\copyrightyear{2024}
\copyrightclause{Copyright for this paper by its authors.
  Use permitted under Creative Commons License Attribution 4.0
  International (CC BY 4.0).}

\conference{BPM 2024: Demos and Resources, September 01-06, 2024, Krakow, PL}

\title{BPMN Analyzer 2.0: Instantaneous, Comprehensible, and Fixable Control Flow Analysis for Realistic BPMN Models}

\author[1]{Tim Kräuter}
[email=tkra@hvl.no]
\author[1]{Patrick Stünkel}
[email=past@hvl.no] % patrick.stuenkel@hvl.no
\author[1]{Adrian Rutle}
[email=aru@hvl.no]
\author[1]{Yngve Lamo}
[email=yla@hvl.no]
\author[2,1]{Harald König}
[email=harald.koenig@fhdw.de]
\address[1]{Western Norway University of Applied Sciences, Bergen, Norway}
\address[2]{FHDW Hannover, Germany}

\begin{abstract}
% 102 words
Many business process models contain control flow errors, such as deadlocks or livelocks, which hinder proper execution.
In this paper, we introduce a new tool that can instantaneously identify control flow errors in BPMN models, make them understandable for modelers, and suggest corrections to resolve them.
We demonstrate that detection is instantaneous by benchmarking our tool against synthetic BPMN models with increasing size and state space complexity, as well as realistic models.
Moreover, the tool directly displays detected errors in the model, including an interactive visualization, and suggests fixes to resolve them.
The tool is open-source, extensible, and integrated into a popular BPMN modeling tool.
\end{abstract}

\begin{keywords}
BPM,
Verification,
Control flow analysis,
BPMN model checking,
Soundness,
Safeness
\end{keywords}

\maketitle

\section{Introduction}

% Problem statement
Business Process Modeling Notation (BPMN) is becoming increasingly popular for automating processes and orchestrating people and systems.
However, many process models suffer from control flow errors, such as deadlocks, livelocks, and starvation~\cite{fahlandAnalysisDemandInstantaneous2011}.
These errors hinder the correct execution of BPMN models and may be detected late in the development process, resulting in elevated costs.

% Solution
In this paper, we describe a new tool, the \textit{BPMN Analyzer 2.0}\footnote{
In the following, we will use BPMN Analyzer to refer to the BPMN Analyzer 2.0, not our previous work in~\cite{krauterHigherorderTransformationApproach2024}.}, for analyzing BPMN process models to detect control flow errors \textit{already} during modeling.
\autoref{fig:overview} shows an overview of the tool.
The tool front-end is based on the popular \textit{bpmn.io} ecosystem, while the analysis is implemented in Rust for performance and memory efficiency reasons.
We implemented a breadth-first state space exploration~\cite{clarkeHandbookModelChecking2018}.
While generating the state space, we check BPMN soundness and safeness~\cite{corradiniClassificationBPMNCollaborations2018} \textit{on-the-fly} to uncover control flow errors.
Consequently, the tool can detect deadlocks, livelocks, starvation, dead activities, and lack of synchronization in BPMN models.
The BPMN Analyzer is open-source and accessible online\footnote{\url{https://timkraeuter.com/bpmn-analyzer-js/}} alongside a video demonstration\footnote{\url{https://www.youtube.com/watch?v=Nv2W-hXNZYA}}~\cite{krauterInstantaneousComprehensibleFixable2024}.

\begin{figure}[ht]
	\centering
	\includegraphics[width=1\linewidth]{images/overview}
	\caption{Overview of the BPMN Analyzer 2.0}
	\label{fig:overview}
\end{figure}

The tool can check models after each change since analysis is \textit{instantaneous} according to~\cite{fahlandAnalysisDemandInstantaneous2011}, i.e., it takes 500ms or less.
Furthermore, we ensure the results are \textit{comprehensible} by highlighting possible violations directly in the model and displaying an interactive counterexample visualization.
Finally, the tool suggests \textit{fixes} for the most common control flow errros and can be extended to suggest more fixes in the future.

% Why is it significant for the BPM field?
Fahland et al.~\cite{fahlandAnalysisDemandInstantaneous2011} describe \textit{coverage}, \textit{immediacy}, and \textit{consumability} as the main challenges for users unaccustomed to formal analysis.
The BPMN Analyzer addresses all these challenges since it supports the most common BPMN elements used in practice (coverage), provides \textit{instantaneous} results (immediacy), and a \textit{comprehensible} user interface (consumability), even including suggestions of fixes.
Developers of industrial BPMN solutions also like our tool, especially the End-2-End user journey~\cite{krauterInstantaneousComprehensibleFixable2024}.
Thus strengthening our claim of a UI that is understandable to users unfamiliar with formal analysis.

% Paper structure
In the remainder of the paper, we describe how instantaneous, comprehensible, and fixable control flow error detection is achieved in \autoref{sec:innovations}.
Then, we discuss the maturity of our tool in \autoref{sec:maturity} before presenting related work in \autoref{sec:related-work}.
Finally, we conclude in \autoref{sec:conclusion}.

% Must have section according to the description
% A section discussing the innovations of the tool or resource to the BPM community and its main characteristics or features
\section{Innovations} \label{sec:innovations} % Main tool description and showing the innovations
The BPMN Analyzer has three main innovations: \textbf{instantaneous}, \textbf{comprehensible}, and \textbf{fixable} control flow error detection.
In this section, we will present the innovations, and more details can be found in our extended paper~\cite{krauterInstantaneousComprehensibleFixable2024}.

\subsection{Instantaneous Analysis}

We demonstrate instantaneous control flow analysis by benchmarking our tool in \textit{three} different scenarios.
For all our benchmarks, we use the hyperfine benchmarking tool \cite{peterHyperfine2023} (version 1.18.0), which calculates the average runtime when executing each control flow analysis ten or more times.
The benchmarks were run on Ubuntu 22.04.4 with an AMD Ryzen 7700X processor (4.5GHz) and 32 GB of RAM (5600 MHz).
All used BPMN models, our tools to generate them, and benchmarking scripts to run them are available in~\cite{krauterInstantaneousComprehensibleFixable2024}.

\textbf{First}, we benchmarked how our tool handles \textbf{BPMN models of growing size}.
We generated 500 synthetic BPMN models starting with five elements up to 4000.
The models repeatedly contain three activities and an exclusive/parallel block with two branches containing one activity per branch (more details in~\cite{krauterInstantaneousComprehensibleFixable2024}).

\begin{figure}[ht]
	\centering
	\includegraphics[width=1\linewidth]{images/model-size-benchmark}
	\caption{Analysis runtime for models of increasing size}
	\label{fig:model-size-benchmark}
\end{figure}

\autoref{fig:model-size-benchmark} shows that the BPMN Analyzer spends from 1 ms up to 9 ms for the BPMN models compared to 0.7 s up to 14 s in our previous tool~\cite{krauterFormalizationAnalysisBPMN2023}.
In summary, the linear growth of the models leads to a linear growth in state space and, consequently, a linear growth in runtime observed in \autoref{fig:model-size-benchmark}.

\textbf{Second}, we benchmarked the tool against a synthetic data set of models that lead to a state space explosion~\cite{clarkeHandbookModelChecking2018}.
This represents a \textit{worst case} scenario for formal analysis.
We generated a data set of models~\cite{krauterInstantaneousComprehensibleFixable2024} with a growing number of parallel branches with increasing length as shown in \autoref{fig:parallel-branches-models}, similar to~\cite{corradiniFormalApproachAnalysis2021}.

\begin{figure}[ht]
	\centering
	\includegraphics[width=1\linewidth]{images/parallel-branches}
	\caption{Models with a growing number of parallel branches and branch length}
	\label{fig:parallel-branches-models}
\end{figure}

\autoref{tab:parallel-branches-benchmark} shows the average runtime of our tool when analyzing these models.
The BPMN Analyzer explores the entire state space while simultaneously analyzing the control flow, i.e., verifying soundness properties.

\begin{table}[h]
	\centering
	\caption{Benchmark results of the parallel branches models}
	\label{tab:parallel-branches-benchmark}
	\SetTblrInner{colsep=2pt}
	\begin{tblr}{
			column{1-X} = {c},
			column{Y-Z} = {r},
			hline{1, 2, Z} = {-}{1.2pt, solid}, % Z is the last row/column
			hline{8, 10} = {-}{dashed},
			vline{2-Y} = {2-Z}{solid}, % Y is the second to last row/column
		}
		\textbf{Branches} & \textbf{Branch Length} &\textbf{Runtime} & \textbf{States} \\
		5 & 1 & 1 ms& 35 \\
		10 & 1 & 3 ms& 1.027 \\
		15 & 1 & 161 ms& 32.771 \\
		16 & 1 & 360 ms& 65.539 \\
		17 & 1 & 790 ms& 131.075 \\
		20 & 1 & 8.803 ms& 1.048.579 \\
		5 & 3 & 3 ms& 1.027 \\
		5 & 5 & 14 ms& 7.779 \\
		3 & 10 & 2 ms& 1.334 \\
		3 & 20 & 11 ms& 9.264 \\
	\end{tblr}
\end{table}

The models' state space grows exponentially, leading to the same order of growth in runtime.
Our analysis is not instantaneous anymore when approaching 17 parallel branches of length 1 (see \autoref{tab:parallel-branches-benchmark}).
However, analysis is still instantaneous for more reasonable models with five parallel branches of length 5 or 3 branches of length 20.
Other tools report 2-3s of runtime for most soundness properties and 30s for a model with five parallel branches~\cite{corradiniFormalApproachAnalysis2021}, which took only milliseconds in our tool.

\textbf{Third}, we applied our tool to eight \textbf{realistic models}, where three models (e001, e002, e020) are taken from~\cite{houhouFirstOrderLogicVerification2022}, and the remaining five models are part of the Camunda BPMN for research repository\footnote{\url{https://github.com/camunda/bpmn-for-research} \label{footnote:camundaResearch}}.
\autoref{tab:realistic-models-benchmark} shows each model's average runtime and number of states.

\begin{table}[h]
	\centering
	\caption{Benchmark results of the realistic BPMN models}
	\label{tab:realistic-models-benchmark}
	\SetTblrInner{colsep=2pt}
	\begin{tblr}{
			column{Y-Z} = {r},
			hline{1, 2, Z} = {-}{1.2pt, solid}, % Z is the last row/column
			hline{5} = {-}{dashed},
			vline{2-Y} = {2-Z}{solid}, % Y is the second to last row/column
		}
		\textbf{Model name} &\textbf{Runtime} & \textbf{States} \\
		e001~\cite{houhouFirstOrderLogicVerification2022} & 1 ms & 39 \\
		e002~\cite{houhouFirstOrderLogicVerification2022} & 1 ms & 39 \\
		e020~\cite{houhouFirstOrderLogicVerification2022} & 10 ms & 5356 \\
		credit-scoring-async\footref{footnote:camundaResearch} & 1 ms & 60 \\
		credit-scoring-sync\footref{footnote:camundaResearch} & 1 ms & 140 \\
		dispatch-of-goods\footref{footnote:camundaResearch} & 1 ms & 103\\
		recourse\footref{footnote:camundaResearch} & 1 ms & 77 \\
		self-service-restaurant\footref{footnote:camundaResearch} & 1 ms & 190 \\
	\end{tblr}
\end{table}

The BPMN Analyzer takes 1-10ms for e001, e002, and e020 ~\cite{krauterInstantaneousComprehensibleFixable2024} while~\cite{houhouFirstOrderLogicVerification2022} and~\cite{krauterFormalizationAnalysisBPMN2023} report 3.66-10.26s and 1-1.75s.
The benchmarks in~\cite{krauterFormalizationAnalysisBPMN2023} were run on the same hardware, while the machine used in~\cite{houhouFirstOrderLogicVerification2022} was slightly less powerful.
To summarize, we believe our analysis is instantaneous for nearly all models since most models have relatively small state spaces (less than 1000 states, see~\cite{fahlandAnalysisDemandInstantaneous2011}).

\subsection{Comprehensible Analysis}
We implemented two features to make control flow analysis understandable for everyone. 
% Highlighting
\textbf{First}, we highlight the problematic elements that cause control flow errors by directly attaching red overlays to them in the BPMN model.
In addition, there is a summary panel in the top-right stating if any errors are found.

% Counterexample
\textbf{Second}, we use \textit{tokens} to \textit{interactively} visualize errors, i.e., we show an execution leading to the error.
Our analysis provides sample executions leading to the possible control-flow errors, which we visualize directly in the BPMN editor by showing how tokens move from the process start to an erroneous state.
We are unaware of other tools that visualize errors directly in the editor and allow for interactions, such as stopping/resuming, restarting, and speeding up the execution.

% Describe the figure.
In \autoref{fig:counterexample}, the visualization has been \textit{paused} just before an \textit{unsafe} state was reached.
One token is already located at the sequence flow, which is marked, while a second token is currently waiting at the exclusive gateway \textsf{e1}.
The visualization can be resumed or restarted using the play and restart button on the left side.
When resumed, the gateway \textsf{e1} will execute, resulting in two tokens at the subsequent sequence flow, i.e., an unsafe execution state.
In addition, one can control the visualization speed using the bottom buttons next to the speedometer.

\begin{figure*}[ht]
	\centering
	\includegraphics[width=0.8\linewidth]{images/counter-example}
	\caption{\textit{Interactive} erroneous execution example shown in the BPMN Analyzer}
	\label{fig:counterexample}
\end{figure*}

\subsection{Fixable Analysis}
In addition to detecting, highlighting, and visualizing control flow errors, the BPMN Analyzer suggests fixes similar to \textit{quick fixes} in Integrated Development Environments.
Quick fixes cannot be provided for all errors, but we currently cover many patterns leading to deadlocks, lack of synchronization, message starvation, and reused end events.
The quick fixes we support are described in detail in~\cite{krauterInstantaneousComprehensibleFixable2024} and can be extended independently of the formal analysis.
We are unaware of other tools that offer fixes for identified control-flow errors.

For example, \autoref{fig:quick-fixes} shows a screenshot of our tool, where quick fixes are depicted as green overlays containing a light bulb icon.
A user can apply a quick fix by clicking on a green overlay and instantly see the changes regarding control flow errors.
If unhappy with the result, a user can undo all changes since each quick fix is entirely revertible due to the command pattern.
A user might not like a quick fix if it not only fixes an error but also has unintended side effects, such as introducing a different control flow error.

\begin{figure*}[ht]
	\centering
	\includegraphics[width=0.8\linewidth]{images/quick-fix}
	\caption{Suggested \textit{quick fixes} in the BPMN Analyzer}
	\label{fig:quick-fixes}
\end{figure*}

\section{Maturity of the tool} \label{sec:maturity}
The BPMN Analyzer is our newest tool, incorporating many findings from our previous work~\cite{krauterFormalizationAnalysisBPMN2023} while focusing on instantaneous and understandable error detection.
The tool is open-source~\cite{krauterInstantaneousComprehensibleFixable2024}, and we aimed for high code quality by employing industry best practices such as rigorous static analysis, testing, and consistent formatting.
Furthermore, we demonstrated the tool to companies in the BPMN process orchestration space and received positive feedback~\cite{krauterInstantaneousComprehensibleFixable2024}.

As described in the previous section, we benchmarked our tool runtime in three scenarios.
The tool can check most models instantaneously, and we provide more details, including scripts and all models needed to reproduce our results in~\cite{krauterInstantaneousComprehensibleFixable2024}.

\section{Related work} \label{sec:related-work}

\textbf{BPMN specification coverage:}
Most related work focuses on the depth of BPMN formalization while providing control flow analysis by checking soundness and safeness.
Thus, these approaches show how different BPMN elements can be formalized and compare themselves with each other regarding supported elements~\cite{krauterHigherorderTransformationApproach2024,krauterFormalizationAnalysisBPMN2023,corradiniFormalApproachAnalysis2021,houhouFirstOrderLogicVerification2022}.
The supported BPMN elements come close to the capabilities of popular process orchestration platforms.
Our tool supports the same depth of BPMN elements as~\cite{corradiniFormalApproachAnalysis2021}, i.e., the most used gateways, tasks, and events.
A detailed comparison is available in~\cite{krauterInstantaneousComprehensibleFixable2024}, and we plan to support more elements in the future.
We focus on tool performance and capabilities concerning comprehension, error resolution, and seamless integration into BPMN modeling tools.

\textbf{Tool performance:}
Comparing tool performance without standard benchmarks and a reproducible environment is challenging.
However, other publications indicate that other tools take several seconds up to half a minute to check single soundness properties~\cite{krauterHigherorderTransformationApproach2024,corradiniFormalApproachAnalysis2021,houhouFirstOrderLogicVerification2022}.
In contrast, our approach instantaneously checks \textit{all} soundness properties and safeness of the same models.
The difference in performance lies probably in our pragmatic BPMN encoding optimized for analysis and its direct implementation in an efficient programming language rather than using general model-checking tools.

\textbf{Tool capabilities:}
Another way to compare the different BPMN formalizations and analysis tools is to investigate their capabilities.
Most tools formalize large parts of the BPMN specification and allow soundness and safeness checking~\cite{krauterHigherorderTransformationApproach2024,krauterFormalizationAnalysisBPMN2023,corradiniFormalApproachAnalysis2021,houhouFirstOrderLogicVerification2022}.
Some tools investigate additional aspects such as the introduction of \textit{data} and \textit{time} during verification~\cite{houhouFirstOrderLogicVerification2022,corradiniFormalisingAnimatingMultiple2022}.
Furthermore, other tools allow specifying and checking \textit{custom temporal logic properties}~\cite{krauterFormalizationAnalysisBPMN2023,corradiniFormalApproachAnalysis2021} and even provide graphical interfaces to ease the specification~\cite{krauterHigherorderTransformationApproach2024}
Moreover, some tools provide interactive BPMN simulation using token-flow animation~\cite{corradiniFormalisingAnimatingMultiple2022,camundaservicesgmbhBpmnjsTokenSimulation2024}, while others \textit{visualize counterexamples} for soundness violations using tokens~\cite{houhouFirstOrderLogicVerification2022}.

Our tool focuses on \textit{instantaneous} control flow analysis and does not support custom properties, data, or time.
We do not provide BPMN simulation since other tools already offer this.
However, we use token-flow animation to visualize control flow errors interactively.
This improves comprehension compared to previous static, less interactive visualizations.
In addition, to the best of our knowledge, our tool is the only one that provides \textit{quick fixes}, i.e., automatic resolutions if control flow errors are detected.
To sum up, our tool incorporates and enhances several ideas from the state of the art while adding novel concepts, such as quick fixes.
We advocate for a pragmatic approach, prioritizing performance and understanding above all else to ensure seamless integration into BPMN modeling tools.

\section{Conclusion \& Future work} \label{sec:conclusion}

In this paper, we describe the novel \textit{BPMN Analyzer} that provides instantaneous, comprehensible, and fixable BPMN control flow error detection and is integrated into a popular BPMN modeling tool.
We benchmarked our tool against synthetic and realistic BPMN models to demonstrate instantaneous soundness checking.
We address the three main challenges: \textit{coverage}, \textit{immediacy}, and \textit{consumability} for providing formal analysis to non-expert users as identified in~\cite{fahlandAnalysisDemandInstantaneous2011}.
In addition, our tool provides quick fixes for common patterns that lead to control flow errors.
One can understand the BPMN Analyzer as a BPMN-specific model checker, implemented in Rust paired with an intuitive user interface based on the popular \textit{bpmn.io} ecosystem that is open for extension by design.

In future work, we aim to improve our tool by providing more quick fixes, considering advanced BPMN elements such as different events, and ranking quick fixes based on their usefulness.
For example, the impact of quick fixes on all control flow errors can be part of the ranking since the resulting model can be checked instantaneously in the background.
Other metrics, such as least change and least surprise from the model repair field, can be used, or one can learn from previous user behavior.
Finally, we aspire to test our tool in a real-world scenario to gather feedback and measure its impact on productivity.

% \begin{acknowledgments}
% Thanks to the anonymous reviewers for their valuable feedback.
% \end{acknowledgments}

\bibliography{bib}

\end{document}
